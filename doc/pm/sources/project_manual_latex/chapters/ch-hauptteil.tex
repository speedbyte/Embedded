%--------------------------------------------
% Chapter: PROJECT GOALS
%--------------------------------------------
\chapter{Project goals}
\label{sec:goals}

\section{Temporal scope and milestones}
\label{sec:goals:scope}
Start of project: March 23, 2015\\
End of project: June 19, 2015

\begin{enumerate}
	\item Milestone: April 1, 2015\\
				-- Hardware selected
	\item Milestone: May 11, 2015\\
				-- HAL drivers finalized
	\item Milestone: May 25, 2015\\
				-- First prototyp with flying capabilities\\
				-- First simulation prototype for autonomous landing
	\item Milestone: June 15, 2015\\
				-- First prototyp with position hold and autonomous landing capabilities
\end{enumerate}

\section{Intended contentual goals}
\label{sec:goals:intention}
\begin{itemize}
	\item Selecting new hardware for an exisiting quadrocopter system
	\item Setting up a Real-Time Operating System (linux-based) for new hardware
	\item Do extensive testing on new sensors (behaviour, sensor models, etc.)
	\item Develop an autonomous landing functionality for the new Quadrocopter platform
	\item Develop a position hold functionality (GPS-hold) for new Quadrocopter platform
\end{itemize}

\section{Contentual framework}
\label{sec:goals:framework}
\begin{itemize}
	\item Select new hardware (Raspberry-based with Linux capability) 
	\item Setting up RTOS (Preempt RT Kernel)
	\item Writing HAL drivers for new Hardware
	\item Writing a sensor fusion for orientation filtering and signal enhancement
	\item Integrate existing control software (if possible)
	\item Setup of sensor models to simulate landing control
	\item Setup of flight model to simulate position hold
\end{itemize}

%--------------------------------------------
% Chapter: PERSONNEL
%--------------------------------------------
\chapter{Personnel}
\label{sec:personnel}

\section{Contact information}
\label{sec:personnel:contact}
\begin{itemize}
	\item \textemphs{Oliver Breuning}\\
				Email: olbrgs00@hs-esslingen.de
	\item \textemphs{Martin Brodbeck}\\
				Email: mabrgs00@hs-esslingen.de
	\item \textemphs{J\"urgen Schmidt}\\
				Email: juscgs00@hs-esslingen.de			 
	\item \textemphs{Philipp Woditsch}\\
				Email: phwogs00@hs-esslingen.de
\end{itemize}

\section{Roles and Responsibilities}
\label{sec:personnel:respons}
\textemphs{Oliver Breuning}
\begin{itemize}
	\item Documentation Quality Manager
	\item Software Developer
\end{itemize}


\textemphs{Martin Brodbeck}
\begin{itemize}
	\item Product Quality Manager
	\item Software Developer
\end{itemize}


\textemphs{J\"urgen Schmidt}
\begin{itemize}
	\item Project Manager
	\item Software Developer
\end{itemize}


\textemphs{Philipp Woditsch}
\begin{itemize}
	\item Software Developer
	\item Testing Quality Manager
\end{itemize}

%--------------------------------------------
% Chapter: WORKINGS
%--------------------------------------------
\chapter{Workings}
\label{sec:work}

\section{Reporting and meetings}
\label{sec:work:report}
There will be a weekly meeting with Prof. Dr. Friedrich to discuss the progress of the project. Progress will be shown by reports (documentation files, stored \/doc\/* in SVN), project plan status (data in track+) and files in the subversion system. Furthermore, whenever the project team will have a internal meeting, a meeting report will be created. This meeting report can be additionally used for status reporting.

A template of a meeting report can be found in the subversion system at
\begin{align*}
\texttt{/doc/pm/meetings/meetingReport\_template.docx} \quad.
\end{align*}A finalized meeting report shall be stored in \texttt{/doc/pm/meetings} in SVN, following the naming convention \textbf{\texttt{meetingReport\_YYYYMMDD.docx}}. If there will be more than one meeting per day, the report file shall be extended. Each report shall be stored as a \textbf{docx-file} and a \textbf{PDF}.

\textbf{Important remark:}\\
The author of the report and all participated team members have to be named!

\section{Documentation}
\label{sec:work:docu}
The project team will deliver the following documents:
\begin{itemize}
	\item Project manual (\texttt{/doc/pm})
	\item Technical drawing of hardware (\texttt{/doc/se})
	\item Bill of materials (\texttt{/doc/se})
	\item Software structure \& concept (\texttt{/doc/se})
	\item Testing concept for sensors (\texttt{/doc/se})
	\item User manual of project (\texttt{/doc/se})
\end{itemize}
All documents will be created during the development. The project team will keep the content of the documentation as close as possible to the actual progress of development.

\section{Team communication}
\label{sec:work:comm}

\subsection{Meetings}
The project team will have \textbf{meeting on a weekly basis}, on fridays. The progress of the current week will be discussed. Problems will be clarified and - if needed - the project plan will be adapted to the latest occurrences.

\subsection{Code reviewing}
Additionally, whenever a task for software implementation will be near to 'completed' the \textbf{source code} will be \textbf{reviewed} by another team member to ensure the software quality and style guide conformity.

\section{Groupware conventions}
\label{sec:work:groupware}

\subsection{Folder structure in SVN}
\label{sec:work:groupware:svn_folders}
For version controlling, the subversion system shall be used. Every team member will regard the following given folder structure in order to keep a structured and organized working flow.
\dirtree{%
.1 /.
.2 doc\DTcomment{\textbf{Finalized documentation}}.
.3 pm\DTcomment{Documentation regarding project management}.
.4 sources\DTcomment{LaTex sources of all documents in \texttt{/doc/pm}}.
.3 se\DTcomment{Documentation regarding system engineering}.
.4 sources\DTcomment{LaTex sources of all documents in \texttt{/doc/se}}.
.2 impl\DTcomment{\textbf{Source code}}.
.3 branch\DTcomment{Container for parallel development lines to trunk}.
.3 tag\DTcomment{Finalized and stable software}.
.3 trunk\DTcomment{Current development line}.
.4 app\DTcomment{Application Layer Software}.
.4 hal\DTcomment{Hardware Abstraction Layer Software}.
.4 sig\DTcomment{Signal Processing Layer Software}.
.2 scratch\DTcomment{\textbf{Documents or files in progress, excluding source code}}.
.3 doc\_template\DTcomment{Templates for documentation files (mainly LaTex)}.
.3 meetings\DTcomment{Meeting reports and meeting template}.
.2 sys\DTcomment{\textbf{Firmwares, OS and development environment}}.
.3 RPi\DTcomment{Preconfigured firmware(s) for Raspberry Pi}.
.3 VM\DTcomment{Development VM for Desktop with all IDEs preconfigured}.
}

\newpage
Within the folders 
\begin{itemize}
	\item \texttt{/impl/trunk/app}
	\item \texttt{/impl/trunk/hal}
	\item \texttt{/impl/trunk/sig}
\end{itemize}
a subfolder for each functional unit shall be created. The functional units shall be considered as depicted in \texttt{/doc/se/software\_concept.pdf}. Example:
\dirtree{%
.1 /.
.2 impl.
.3 trunk.
.4 hal.
.5 adc.
.5 gps.
.5 gyro.
}

\subsection{Storing test files}
\label{sec:work:groupware:test}
For each software component, a separate test should be written to ensure the software quality. All source code files and test data files to run a test shall be saved in a separate subfolder named \texttt{tst}. Example:
\dirtree{%
.1 /.
.2 impl.
.3 trunk.
.4 hal.
.5 gps.
.6 tst\DTcomment{contains data \& code to test the module 'gps'}.
.7 test\_data.dump.
.7 helper\_code.c.
.7 helper\_code.h.
.7 test\_main.c.
.6 gps.c\DTcomment{actual source of module 'gps'}.
.6 gps.h\DTcomment{actual interfaces of module 'gps'}.
}


\textbf{Important note:}\\
The \texttt{tst}-folder will \textbf{not} be moved to the tags folders! All data in \texttt{/imp/tags} is considered to be well tested and stable. Therefore, the test data is not required to move.