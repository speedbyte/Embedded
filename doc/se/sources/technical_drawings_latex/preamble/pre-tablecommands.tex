% ------------------------------------------------------------------------
% LaTeX - Preambel ******************************************************
% ------------------------------------------------------------------------
% Table Commands
% ------------------------------------------------------------------------
% basiernd auf www.matthiaspospiech.de/latex/vorlagen Diplomarbeit kompakt
% ========================================================================
%% Kommandos fuer Tabellen. Entnommen aus The LateX Companion, tabsatz.ps und diversen Dokus

%%% ---| Farben fuer Tabellen |-------------------
\colorlet{tablesubheadcolor}{gray!30}
\colorlet{tableheadcolor}{gray!25}
\colorlet{tableblackheadcolor}{black!100}
\colorlet{tablerowcolor}{gray!10.0}
%%% ---------------------------------------------

% um Tabellenspalten mit Flattersatz zu setzen, muss \\ vor
% (z.B.) \raggedright geschuetzt werden:
\newcommand{\PreserveBackslash}[1]{\let\temp=\\#1\let\\=\temp}

% Linksbuendig:
\newcolumntype{v}[1]{>{\PreserveBackslash\RaggedRight\hspace{0pt}}p{#1}}
\newcolumntype{M}[1]{>{\PreserveBackslash\RaggedRight\hspace{0pt}}m{#1}}
\newcolumntype{Y}{>{\PreserveBackslash\RaggedLeft\hspace{0pt}}X}

\newcolumntype{Z}{>{\PreserveBackslash\RaggedRight\hspace{0pt}}X}

%%% ---|Layout der Tabellen |-------------------


% Groesse der Schrift in Tabellen
\newcommand{\tablefontsize}{ \footnotesize}
\newcommand{\tableheadfontsize}{\footnotesize}

% Layout der Tabelle: Ausrichtung, Schrift, Zeilenabstand
\newcommand\tablestylecommon{%
  \renewcommand{\arraystretch}{1.4} % Groessere Abstaende zwischen Zeilen
  \normalfont\normalsize            %
  \sffamily\tablefontsize           % Serifenlose und kleine Schrift
  \centering%                       % Tabelle zentrieren
}

\newcommand{\tablestyle}{
	\tablestylecommon
	%\tablealtcolored
}

% Ruecksetzten der Aenderungen
\newcommand\tablerestoresettings{%
  \renewcommand{\arraystretch}{1}% Abstaende wieder zuruecksetzen
  \normalsize\rmfamily % Schrift wieder zuruecksetzen
}

% Tabellenkopf: Serifenlos+fett+schraeg+Schriftfarbe
\newcommand\tablehead{%
  \tableheadfontsize%
  \sffamily\bfseries%
  %\slshape
  %\color{white}
}

\newcommand\tablesubheadfont{%
  \tableheadfontsize%
  \sffamily\bfseries%
  \slshape
  %\color{white}
}


\newcommand\tableheadcolor{%
	%\rowcolor{tablesubheadcolor}
	%\rowcolor{tableblackheadcolor}
	\rowcolor{tableheadcolor}%
}

\newcommand\tablesubheadcolor{%
	\rowcolor{tablesubheadcolor}
	%\rowcolor{tableblackheadcolor}
}

\newcommand{\tableend}{\arrayrulecolor{black}\hline}


\newcommand{\tablesubhead}[2]{%
  \multicolumn{#1}{>{\columncolor{tablesubheadcolor}}l}{\tablesubheadfont #2}%
}

% Tabellenbody (=Inhalt)
\newcommand\tablebody{%
\tablefontsize\sffamily\upshape%
}

\newcommand\tableheadshaded{%
	\rowcolor{tableheadcolor}%
}
\newcommand\tablealtcolored{%
	\rowcolors{1}{tablerowcolor}{white!100}%
}
%%% --------------------------------------------
