% Bsp. eines Hauptteils

\chapter{Landing Concept}
\label{cha:LandingCon}


This Chapter is about the (theoretical) ways, how to manage a autonomous landing.

In anyway we need to adjust the speed of the rotors, depending on our measurements.


\section{Challenges}
\label{sec:landcha}

There will be several critical "challenges" during a landing process. Some of them when can estimate before and give a option, how to handle them.


\subsection{Drift}

It is very hard, to keep a Quadrocopter (in our case HElikopter) on a fixed point in a room. There is always a little bit of wind, at least from the Quadrocopter itself. Therefore you always need to correct you position, to avoid it drifting away to one side.\\
With our sensors implemented we can recognize the movement and take some countermeasures.\\

The GPS system won't work indoor, so we need to rely on the IMU (Inertial Measurement Unit). So there is no absolute measurement of the position and we need to do some calculation with the acceleration in XY-Axis, to handle the drift.\\

With the gyroscope we can also handle the drift about our own axis.


\subsection{Orientation}

Because of our limited sensors, and a non working GPS indoor, it will not be possible to get a absolute orientation in a room.\\

Outdoors we can take the GPS with the barometer, the gyroscope and the laser sensor to get a very good knowing of where we are.


\subsection{Distance}

In a steady flight situation, we measure the distance straight to the ground. But when the quadrocopter is moving, we need to calculate the distance to ground with the angle of it.


\subsection{Landing Zone}

The first step will be, to take control of the landing watching with our eyes. Just decreasing the speed of the motors (rotors) depending on the distance to ground.\\

When we try that the Quadrocopter takes care of a free landing zone by itself, we can think about different options:\\

\underline{Circling}

With this method we just fly over the whole landing zone, which is a big as the quadrocopter. When we got the same distance value for the whole area, we can assume that there is no obstacle within it.


\underline{Remembering}

If we take some measurements during normal flight mode, we can save that values and try to build a map of the area under us. With this method we can fly to a "free" landing zone, when the landing mode is enabled.

\underline{lateral buckling}

May be the fastest and the most difficult way to land. By lateral buckling/tilting and calculation of the gathered values (depending on the angle), we can get a fast look over the area under us.

The challenge will be to stay, more or less, on the same position. Considering a certain height, where doing this action, there may be needed only small angles to tilt.

Or we could think about, mounting the laser sensor with a small angle. So to speak with a offset. We can easily consider the angle in the calculation of the height and only need to turn on the quadrocopters z-axis to check the landing zone.

\newpage
\section{Landing Mode}

The Landing mode should handle the landing of the quadrocopter fully autonomous.

\subsection{Activation}

The Landing Mode will/must be activated external. With it some sensor could be activated too, or configured in a other way. 


\subsection{Abort Landing}

We should be able to abort the autonomous landing, if we detect a problem.

1. approach: Just turning the Landing Mode off.

2. approach: Override the Landing Mode with the remote control.

3. approach: Abort the landing and turn the rotors off, to prevent damage to them.






%Verweise im Text: \cite{doc:stz} und \cite{doc:gun}.

\chapter{Open Points}
\label{sec:ergeb}

There are several points we wanted to do, but could not manage to do or finish them in time.

\section{Connecting actuators}

We did not connect any actuator, like an rotor, to our current software. There was no actuator-software ported to our Raspberry Pi to test with our developments.



\section{Reacting on sensor values}

There was no development to react, in any way, on our provided sensor values.\\
For this reason there is no controller yet, processing external signals or work between sensor and actuator.


%\enquote{Neuigkeiten} Messergebnisse



